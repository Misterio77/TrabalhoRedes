\documentclass[12pt]{article}
\usepackage{sbc-template}
\usepackage[brazil]{babel}
\usepackage[utf8]{inputenc}
\usepackage{indentfirst}
\usepackage{listings}
\usepackage{xcolor}
\usepackage{url}

\definecolor{codegreen}{rgb}{0,0.6,0}
\definecolor{codegray}{rgb}{0.5,0.5,0.5}
\definecolor{codepurple}{rgb}{0.58,0,0.82}
\definecolor{backcolour}{rgb}{0.95,0.95,0.92}

\lstdefinestyle{mystyle}{
    backgroundcolor=\color{backcolour},   
    commentstyle=\color{codegreen},
    keywordstyle=\color{magenta},
    numberstyle=\tiny\color{codegray},
    stringstyle=\color{codepurple},
    basicstyle=\ttfamily\footnotesize,
    breakatwhitespace=false,         
    breaklines=true,                 
    captionpos=b,                    
    keepspaces=true,                 
    numbers=left,                    
    numbersep=5pt,                  
    showspaces=false,                
    showstringspaces=false,
    showtabs=false,                  
    tabsize=2
}

\lstset{style=mystyle}

\title{Ensino de redes com ferramentas "as-code"}
\author{Gabriel Silva Fontes\inst{1} \\ g.fontes@usp.br}

\vbadness=10000
\hbadness=10000

\address{Instituto de Ciências Matemáticas e de Computação -- Universidade de São Paulo (ICMC/USP)}

\begin{document}
\maketitle

\begin{resumo}
    O uso de Infrastructure-as-code (IaC) na indústria é crescente, e suas vantagens no processo de desenvolvimento de infraestruturas são frequentemente citadas como um pilar do desenvolvimento em cloud moderno. Esse trabalho busca entender o atual interesse e uso de ferramentas as-code no ensino de computação, e o desenvolvimento de uma prova de conceito em cima de uma ferramenta real.
\end{resumo}

\section{Introdução}

Com o crescente uso de nuvens computacionais, onde pode-se criar dispositivos
de rede e infraestrutura programaticamente, há uma proliferação de diferentes
ferramentas e abordagens para provisionar e gerenciar esses recursos.

Andando lado-a-lado com metodologias que priorizam automação e
reprodutibilidade, como DevOps, há o crescente uso de ferramentas de
"Infrastructure as Code" (IaC), que permitem gerir recursos computacionais
(redes, VMs, etc) por meio de código (seja ele declarativo ou imperativo),
trazendo consigo os vários avanços da área de engenharia de software:
versionamento, teste automático, etc.

As aplicações no ensino de computação podem ser bastante relevantes,
estimulando a colaboração e o reuso de soluções.

\section{Metodologia}

Esse trabalho foi conduzido com uma revisão da literatura, e a implementação de
uma PoC, ilustrando o modelo declarativo.

\subsection{Revisão da Literatura}

Conduzi uma busca via Google Scholar com os seguintes termos

\begin{itemize}
    \item "Teaching Infrastructure as Code"
    \item "Teaching Software Defined Networking"
\end{itemize}

A primeira string de busca não foi suficiente para encontrar muitos resultados
relevantes, então resolvi usar Software-defined Networking (SDN) como um
possível sinônimo. Apesar da diferença do propósito (normalmente SDN é usado
para redes dinâmicas\cite{vsuh2017designing}), é possível aprender com o uso de
SDN feito nesses outros trabalhos, e como essas ferramentas foram postas à
prova no ensino.

Salib et. al\cite{salib2018hands} já trabalharam no tópico do ensino de redes
na graduação usando SDN. As ferramentas exploradas foram o
Mininet\cite{mininet} e o GNS3\cite{gns3api}.



\subsection{Prova de Conceito}

Com base na literatura e na experiência do autor, optamos por utilizar o
GNS3\cite{gns3api}. Essa ferramenta livre é similar a outros simuladores de
rede, mas com os diferenciais de ser facilmente hospedável para múltiplos
usuários, e contando com uma API moderna, facilmente acessível via web. Com
ela, podemos construir uma abstração declarativa.

Para validar a viabilidade de interagir programaticamente com o GNS3, montamos
uma prova de conceito (PoC). Essa implementação é um código simples em Python,
que interage com o API do GNS3\cite{gns3api}.

São expostas algumas funções que permitem gerir e configurar nós (e.g.
roteadores). A implementação atual, apesar de declarativa, é rudimentar: é
necessário destruir e recriar o nó a cada mudança, já que não há \textit{drift
detection} implementado.

Dito isso, é possível criar nós, e rodar comandos via telnet, de forma
declarativa. Isso permite criar redes arbitrárias, que podem ser aplicadas com
apenas um comando:

\begin{lstlisting}[language=Python]
project = get_project(SERVER, "pocpoc")

router1 = get_node(SERVER, project, "R1", {
    "compute_id": "local",
    "node_type": "dynamips",
    "x": 0,
    "properties": {
        "platform": "c3745",
        "image": "c3745-adventerprisek9-mz.124-25d.image",
        "nvram": 256,
        "ram": 256, 
        "slot0": "GT96100-FE",
        "slot1": "NM-1FE-TX",
        "slot2": "NM-4T",
        "system_id": "FTX0945W0MY",
        "startup_config_content": """
            hostname ciclano
        """,
    }
}, destroy=True)
start_node(SERVER, project, router1)

router2 = get_node(SERVER, project, "R2", {
    "compute_id": "local",
    "node_type": "dynamips",
    "x": 100,
    "properties": {
        "platform": "c3745",
        "image": "c3745-adventerprisek9-mz.124-25d.image",
        "nvram": 256,
        "ram": 256, 
        "slot0": "GT96100-FE",
        "slot1": "NM-1FE-TX",
        "slot2": "NM-4T",
        "system_id": "FTX0945W0MY",
        "startup_config_content": """
            hostname fulano
        """,
    }
}, destroy=True)
start_node(SERVER, project, router2)

run_node_command(SERVER, project, router2, """
    enable
    conf t
    hostname foobar
    exit
    exit
""")

link_nodes(SERVER, project, router1, router2)
\end{lstlisting}

\section{Resultados Preliminares}

O uso do GNS3 como ferramenta de simulação e ensino parece bastante viável.
Além de poder emular dispositivos cisco (via Dynamips), ele também suporta VMs,
contêineres, e permite colaboração multiusuário (via sua arquitetura de
cliente/servidor, que inclui até uma interface web).

O API do GNS3 é extremamente poderoso, e permite implementar interfaces
declarativas com certa facilidade. É possível criar templates (e.g.
dispositivo), que permitem reuso de configuração. É possível executar comandos
(via telnet), sendo, então, possível configurar qualquer topologia arbitrária
via código.

\section{Trabalhos Futuros}

Para ter o valor esperado, essa PoC deve ser validada, de forma sistemática,
com alunos reais. Deve-se comparar a produtividade do ensino com ferramentas
tradicionais, e montar questionários para levantar vantagens e desvantagens, na
visão dos alunos.

O autor propõe um amadurecimento da PoC. Implementando um API ergonômico e,
possivelmente, drift detection.

Para uma PoC ainda mais madura, o autor propõe a implementaçao de um
\textit{Terraform provider} para interagir com o GNS3; permitindo colher as
vantagens de do Terraform (recursos 100\% declarativos, ótima usabilidade,
drift detection automático, etc).

\bibliographystyle{plain}
\bibliography{references}

\end{document}
