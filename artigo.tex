\documentclass[12pt]{article}
\usepackage{sbc-template}
\usepackage[brazil]{babel}
\usepackage[utf8]{inputenc}
\usepackage{indentfirst}

\title{Ensino de redes com ferramentas "as-code"}
\author{Gabriel Silva Fontes\inst{1} \\ g.fontes@usp.br}

\vbadness=10000
\hbadness=10000

\address{Instituto de Ciências Matemáticas e de Computação -- Universidade de São Paulo (ICMC/USP)}

\begin{document}
\maketitle

\begin{resumo}
    TODO
\end{resumo}

\section{Introdução}

Com o crescente uso de nuvens computacionais, onde pode-se criar dispositivos
de rede e infraestrutura programaticamente, há uma proliferação de diferentes
ferramentas e abordagens para provisionar e gerenciar esses recursos.

Andando lado-a-lado com metodologias que priorizam automação e
reprodutibilidade, como DevOps, há o crescente uso de ferramentas de
"Infrastructure as Code" (IaC), que permitem gerir recursos computacionais
(redes, VMs, etc) por meio de código (seja ele declarativo ou imperativo),
trazendo consigo os vários avanços da área de engenharia de software:
versionamento, teste automático, etc.

\section{Metodologia}

TODO

\subsection{Revisão da Literatura}

Conduzi uma busca via Google Scholar com os seguintes termos

\begin{itemize}
    \item "Teaching Infrastructure as Code"
    \item "Teaching Software Defined Networking"
\end{itemize}

TODO

\subsection{Prova de Conceito}

TODO

\section{Resultados Preliminares}

O uso do GNS3 como ferramenta de simulação e ensino parece bastante viável.
Além de poder emular dispositivos cisco (via Dynamips), ele também suporta VMs,
contêineres, e permite colaboração multiusuário (via sua arquitetura de
cliente/servidor, que inclui até uma interface web).

TODO

\section{Trabalhos Futuros}

Para ter o valor esperado, essa PoC deve ser validada, de forma sistemática,
com alunos reais. Deve-se comparar a produtividade do ensino com ferramentas
tradicionais, e montar questionários para levantar vantagens e desvantagens, na
visão dos alunos.

Para uma PoC mais madura, o autor propõe a implementaçao de um
\textit{Terraform provider} para interagir com o GNS3; permitindo colher as
vantagens de do Terraform (recursos declarativos, ótima usabilidade, etc).

\section{Discussão}

TODO

\bibliographystyle{sbc}
\bibliographystyle{references}

\end{document}
